\section{Tarea}
\begin{enumerate}[{Ejercicio} 1.]
	\item Utilizar el tipo generico de Lista Enlazada para generar los peores casos y ejecutar el
	algoritmo de ordenamiento.\\
	
	\begin{lstlisting}[language=java, caption={caption}][H]
		//List - Method get()
		public Node<T> get(int indice) {
			Node<T> aux=raiz;
			for(int i=0;i<indice;i++)
			aux=aux.getNextNode();
			return aux;
		}
		//List - Method remove()
		public void remove(int indice) {
			if(indice<tamano) {
				if(indice==0)
				raiz=raiz.getNextNode();
				else {
					Node<T> anterior=this.get(indice-1);
					anterior.setNextNode(this.get(indice+1));
				}
				tamano--;
			}
		}
	\end{lstlisting}
	
	
	\item Utilizar el tipo generico de Doble Lista Enlazada para generar los peores casos y ejecutar
	el algoritmo de ordenamiento.
	
	\begin{lstlisting}[language=java, caption={caption}][H]
		public class Node <E > {
			private E data ;
			private Node <E > nextNode ;
			private Node <E > previousNode ;
			
			/* Getters y Setter ... */
			
			Node () {
				this . data = null ;
				this . nextNode = null ;
				this . previousNode = null ;
			}
			Node ( E data ) {
				this . data = data ;
				this . nextNode = null ;
				this . previousNode = null ;
			}
			Node ( E data , Node <E > nextNode ) {
				this . data = data ;
				this . nextNode = nextNode ;
				this . previousNode = null ;
			}
			Node ( E data , Node <E > nextNode , Node <E > previousNode ) {
				this . data = data ;
				this . nextNode = nextNode ;
				this . previousNode = previousNode ;
			}
		}
	\end{lstlisting}
	
\end{enumerate}